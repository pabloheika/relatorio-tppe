% ----------------------------------------------------------
\chapter{Introdução}
\label{cap:intr}
% ----------------------------------------------------------

Este documento e seu código fonte exemplificam a elaboração de trabalho acadêmico (trabalho de conclusão de curso, dissertação e tese) a partir da classe UnB\TeX, uma extensão da classe \abnTeX\  \cite{Castro2019} para a Universidade de Brasília (UnB).

% Definição da nomenclatura que irá para a lista de siglas e abreviações
\nomenclature[A]{UnB}{Universidade de Brasília}

O \abnTeX, por sua vez, é uma customização da classe \textsf{memoir} para atender requisitos da norma ABNT NBR 14724:2011 \emph{Informação e documentação -- Trabalhos acadêmicos -- Apresentação}. Informações sobre esta classe estão reunidas em \url{https://www.abntex.net.br/}.

\nomenclature[A]{ABNT}{Associação Brasileira de Normas Técnicas}

A classe UnB\TeX\ também contempla atualizações mais recentes das normas NBR 6023 \cite{NBR6023:2018} e NBR 10520 \cite{NBR10520:2023} da ABNT, não consideradas no \abnTeX. Alguns dos recursos apresentados na classe UnB\TeX\ baseia-se em soluções adotadas por \citeonline{Castro2019} para editoração dos livros da série \emph{Ensino de graduação} da Editora UnB.

Este documento deve ser utilizado como complemento do manual do \abnTeX\ \cite{abntex2classe} e da classe \textsf{memoir} \cite{memoir}. Mais referências sobre o \LaTeX\ e sobre o \abnTeX\ podem ser obtidas em \url{https://github.com/abntex/abntex2/wiki/Referencias}.

%\begin{mdframed}[style=defnSty,innertopmargin=8pt] % azul
\begin{mdframed}[style=plainSty,innertopmargin=8pt] % verde
{\center \textsc{Texto motivador} \par}
\noindent Esperamos que o UnB\TeX\ aprimore a qualidade do trabalho que você produzirá, de modo que o principal esforço seja concentrado no principal: na contribuição científica.
\end{mdframed}